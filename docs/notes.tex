\documentclass[a4paper,11pt]{article}
\usepackage[T1]{fontenc}
\usepackage[utf8]{inputenc}
\usepackage{lmodern}
\usepackage{verbatim}

\title{Linear programming notes}
\author{Marco Marini}

\begin{document}

\maketitle
\tableofcontents

\begin{abstract}
This document contains notes about Linear programming.
\end{abstract}

\section{Supply chain model}

We define a simplified supply chain model as a system
that produces products with a chain of product transformations performed by suppliers.

Let be
\begin{description}
	\item [$ A = a_1 \dots a_n $]
		the set of supplier types
	\item [$ B = b_1 \dots b_m $]
		the set of product types.
	\item [$  N_i , \;i \in A $]
		the number of suppliers of type $ i $.
\end{description}

The supplier can perform only a single transformation for the its duration.

Let us define
\begin{description}
	\item[ $ N_{i} $ ]
		the number of suppliers $ i \in A $
	\item[ $ V_{i} $ ]
		the value of product $ i \in B $
	\item[ $ Q_{ij} $ ]
		the quantity of produced product $ j \in B  $ for supplier $ i \in A, j $ in production interval
	\item[ $ T_{ij}, i \in A, j \in B $ ]
		the production interval for product $ j \in B  $ in supplier $ i \in A, j $
	\item[$ C_{ijk} $ ]
		the quantity of product $ k \in B $ consumed to produce the product $ j \in Bj $ in production interval of supplier $ i \in A $.
\end{description}

\subsection{Production}

By now we do not consider the constraint on the availability of consuming products. It will be considered later.

When a supplier of type $ i $ is ready to produce we assign a production slot for the product $ j $ such that
\begin{description}
	\item[ $ Y_{ij} $ ] is the weighted distribution of slots
\end{description}

The value $ Y_{ij} $ is constrained to 
\begin{equation}
\label{equ:YConstraints}
	0 \le Y_{ij} \le 1
\end{equation}

Moreover we may put the supplier in idle state for a while
\begin{description}
	\item[$ Z_{i0} $]
		is the idle time during a whole production cycle
\end{description}
with
\begin{equation}
\label{equ:ZConstraints}
	Z_{i0} \ge 0
\end{equation}


The total time of production cycle for supplier $ i $ is
	\[ Z_i = \sum_{j \in B} Y_{ij} T_{ij} + Z_{i0} \]

Let normalize the total time production to 1
\begin{equation}
\label{equ:normalInterval}
	Z_i = \sum_{j \in B} Y_{ij} T_{ij} + Z_{i0} = 1
\end{equation}

During this interval all the suppliers $ i $ produce the product $ j $ at a slot rate
\begin{equation}
\label{equ:prodFreq}
\left.
	S_{ij} =  N_i \frac{Y_{ij}}{Z_i} = N_i Y_{ij}
\right| _{i \in A, j \in B}
\end{equation}

The production rate of product $ j $ for supplier $ i $ is
\[
\left.
	P_{ij} =  Q_{ij} S_{ij} = N_i Q_{ij} Y_{ij}
\right| _{i \in A, j \in B}
\]

While the production rate of product $ j $ is
	\[ P_{j \in B} = \sum_{i \in A,} P_{ij} \]

\subsection{Constraints}

The (\ref{equ:prodFreq}) expresses the slot rate of a product for specific suppliers.

Therefore we can calculate the consumption rate of a product $k$ to produce product $ j $ for suppliers $ i $
\[
\left.
	E_{ijk}  = C_{ijk} S_{ij}
\right| _{i \in A, j,k \in B}
\]

Then the total consumption rate for product $ k $ is
	\[ E_{k} = \sum_{i \in A, j \in B} E_{ijk} \]

If
	\[ E_i < P_i \]
the product s produced at a higher rate then it is consumped creating a surplus that can be sold.

On the other hane we cannot have 
	\[ E_i > P_i \]
because it cannot consume more product than the produced.

So the system must satisfy the constraint
\begin{equation}
\label{equ:consumptionConstraint}
	P_i - E_i \ge 0
\end{equation}
	
\subsection{Value rate}
We can compute the value rate for the whole supply chain
\begin{equation}
\label{equ:valueRate}
	V = \sum_{i \in B}( P_i - E_i ) V_i
\end{equation}

The problem is to find the optimal production configuration that maximize the value rate. This is defined by the linear system composed by
(\ref{equ:YConstraints}),
(\ref{equ:ZConstraints}),
(\ref{equ:normalInterval}),
(\ref{equ:consumptionConstraint}),
(\ref{equ:valueRate})
\[
\left\{
\begin{array}{ll}
	\max_{(Y_{ij}, Z_{i0})} (V) & , i \in A, j \in B \\
	\sum_{j \in B} Y_{ij} T_{ij} + Z_{i0} = 1 & , i \in A \\
	Z_{i0} \ge 0 & , i \in A \\
	Y_{ij} \ge 0 & , i \in A, j \in B \\
	Y_{ij} \le 1 & , i \in A, j \in B \\
	P_i - E_i \ge 0 & , i \in B
\end{array}
\right.
\]

\subsection{Tensor form}

Let us normalize the problem such that
\[
\left\{
\begin{array}{ll}
	\min C^T X \\
	A_1 X = B_1 \\
	A_2 X <= B_2 \\
	X >= 0
\end{array}
\right.
\]
so
\[
	\min (-V) = 
	\min (-\sum_{i \in B}( P_i - E_i ) V_i) = 
\]



\end{document}