\documentclass[a4paper,11pt]{article}
\usepackage[T1]{fontenc}
\usepackage[utf8]{inputenc}
\usepackage{lmodern}

\title{Linear programming notes}
\author{Marco Marini}

\begin{document}

\maketitle
\tableofcontents

\begin{abstract}
This document contains notes about Linear programming.
\end{abstract}

\section{Supply chain model}

Let be define a simplified supply chain model as a system
that produces products with a chain of product transformations performed by producers.

Let be
	\[ A = a_1 \dots a_n \]
the set of producer types and
	\[ B = b_1 \dots b_n \]
the set of product types.

For each producer type $ i $ let be
	\[ N_i , \;i \in A\]
the number of producers of that type.

The system is constrained by some rules:
\begin{enumerate}
	\item the producer can perform only transformations
	matching the producer type.
	\item the producer can perform only a single
	transformation for the its duration.
\end{enumerate}

For each product $ i $ let define the production rule
\[ R_i = (P_i, V_i, Q_i, T_i, C_{i,i}) \]
with
\begin{itemize}
	\item the producer,
	\item the value of product
	\item the quantity of outcome in duration interval
	\item the duration interval
	\item the quantity of product $ j $ consumed to produce the product $ i $  in duration interval.
\end{itemize}

\end{document}