\documentclass[a4paper,11pt]{article}
\usepackage[T1]{fontenc}
\usepackage[utf8]{inputenc}
\usepackage{lmodern}
\usepackage{verbatim}

\title{Linear programming notes}
\author{Marco Marini}

\begin{document}

\maketitle
\tableofcontents

\begin{abstract}
This document contains notes about Linear programming.
\end{abstract}

\section{Supply chain model}

We define a simplified supply chain model as a system
that produces products with a chain of product transformations performed by suppliers.

Let be
\begin{description}
	\item [$ A = a_1 \dots a_n $]
		the set of supplier types
	\item [$ B = b_1 \dots b_m $]
		the set of product types.
\end{description}

The supplier can perform only a single transformation for the its duration.

Let us define
\begin{description}
	\item [$ n_i $]
		the number of suppliers of type $ i \in A $.
	\item[ $ v_{i} $ ]
	the value of product $ i \in B $
	\item[ $ q_{ij} $ ]
		the quantity of produced product $ j \in B  $ for supplier $ i \in A, j $ in production interval
	\item[ $ t_{ij}, i \in A, j \in B $ ]
		the production interval for product $ j \in B  $ in supplier $ i \in A, j $
	\item[$ c_{ijk} $ ]
		the quantity of product $ k \in B $ consumed to produce the product $ j \in Bj $ in production interval of supplier $ i \in A $.
\end{description}

\subsection{Production}

By now we do not consider the constraint on the availability of consuming products. It will be considered later.

When a supplier of type $ i $ is ready to produce we assign a production slot for the product $ j $ such that
\begin{description}
	\item[ $ y_{ij} $ ] is the weighted distribution of slots
\end{description}

The value $ y_{ij} $ is constrained to 
\begin{equation}
\label{equ:YConstraints}
	0 \le y_{ij} \le 1
\end{equation}

Moreover we may put the supplier in idle state for a while
\begin{description}
	\item[$ z_{i0} $]
		is the idle time during a whole production cycle
\end{description}
with
\begin{equation}
\label{equ:ZConstraints}
	z_{i0} \ge 0
\end{equation}


The total time of production cycle for supplier $ i $ is
	\[ z_i = \sum_{j \in B} y_{ij} t_{ij} + z_{i0} \]

Let normalize the total time production to 1
\begin{equation}
\label{equ:normalInterval}
	z_i = \sum_{j \in B} y_{ij} t_{ij} + z_{i0} = 1
\end{equation}

During this interval all the suppliers $ i $ produce the product $ j $ at a slot rate
\begin{equation}
\label{equ:prodFreq}
\left.
	s_{ij} =  n_i \frac{y_{ij}}{z_i} = n_i y_{ij}
\right| _{i \in A, j \in B}
\end{equation}

The production rate of product $ j $ for supplier $ i $ is
\[
\left.
	p_{ij} = q_{ij} s_{ij} = n_i q_{ij} y_{ij}
\right| _{i \in A, j \in B}
\]

While the production rate of product $ j $ is
\begin{equation}
\label{equ:prodRate}
	p_{j \in B} = \sum_{i \in A,} p_{ij} = 
		\sum_{i \in A,} n_i q_{ij} y_{ij}
\end{equation}


\subsection{Constraints}

The (\ref{equ:prodFreq}) expresses the slot rate of a product for specific suppliers.

Therefore we can calculate the consumption rate of a product $k$ to produce product $ j $ for suppliers $ i $
\[
\left.
	e_{ijk}  = c_{ijk} s_{ij}
\right| _{i \in A, j,k \in B}
\]

Then the total consumption rate for product $ k $ is
\begin{equation}
\label{equ:consumRate}
	e_{k} = \sum_{i \in A, j \in B} e_{ijk}
	= \sum_{i \in A, j \in B} n_i c_{ijk} y_{ij}
\end{equation}

If
	\[ e_i < p_i \]
the product s produced at a higher rate then it is consumped creating a surplus that can be sold.

On the other hane we cannot have 
	\[ e_i > p_i \]
because it cannot consume more product than the produced.

So the system must satisfy the constraint
\begin{equation}
\label{equ:consumptionConstraint}
	p_i - e_i \ge 0
\end{equation}
	
\subsection{Value rate}
We can compute the value rate for the whole supply chain
\begin{equation}
\label{equ:valueRate}
	v = \sum_{i \in B}( p_i - e_i ) v_i
\end{equation}

The problem is to find the optimal production configuration that maximize the value rate. This is defined by the linear system composed by
(\ref{equ:YConstraints}),
(\ref{equ:ZConstraints}),
(\ref{equ:normalInterval}),
(\ref{equ:consumptionConstraint}),
(\ref{equ:valueRate})
\[
\left\{
\begin{array}{ll}
	\max_{(y_{ij}, z_{i0})} (v) & , i \in A, j \in B \\
	\sum_{j \in B} y_{ij} t_{ij} + z_{i0} = 1 & , i \in A \\
	z_{i0} \ge 0 & , i \in A \\
	y_{ij} \ge 0 & , i \in A, j \in B \\
	y_{ij} \le 1 & , i \in A, j \in B \\
	p_i - e_i \ge 0 & , i \in B
\end{array}
\right.
\]

\subsection{Tensor form}

Trasformiamo ora la (\ref{equ:prodRate}) in forma tensoriale:
\[
	P_i = A_{ijk} Y_{kj} = A Y
\]
che deve essere equivalente a
\begin{equation}
\label{equ:p_j}
	p_{j'} = \sum_{i'} n_{i'} q_{i'j'} y_{i'j'}
\end{equation}
da cui
\begin{equation}
\label{equ:Aijk}
\begin{array}{ll}
	A_{iik} = n_k q_{ki} \\
	A_{ijk} = 0 &, i \ne j
\end{array}
\end{equation}

Se poniamo
\[
\begin{array}{ll}
	Q_{ij} = q_{ij} \\
	N_{ii} = n_i \\
	N_{ij} = 0, i \ne j
\end{array}
\]
possiamo scrivere
\[
	B_{ij} = N_{ik} Q_{kj} = N Q = n_{i} q_{ij}
\]
quindi
\[
\begin{array}{ll}
	A_{iik} = B_{ki}\\
	A_{ijk} = 0, i \ne j
\end{array}
\]

Poniamo
\begin{equation}
\label{equ:NQI}
	A_{ijk} = \delta_{ijl} B_{kl} = \delta B^T = B \delta =
		N Q \delta = N_{il} Q_{lm} \delta_{mjk}
\end{equation}
dove
\[
\begin{array}{ll}
	\delta_{ijl} = 1 & , i = j = l \\
	\delta_{ijl} = 0
\end{array}
\]
quindi
\[
	P_i = N Q \delta Y
\]

Allo stesso modo trasformiamo la (\ref{equ:consumRate})
\[
	E_i = M_{ijk} Y_{kj} = M Y
\]
che deve essere equivalente a
\begin{equation}
\label{equ:e_k}
	e_{k'} = \sum_{i', j'} n_{i'} c_{i'j'k'} y_{i'j'}
\end{equation}
da cui
\begin{equation}
\label{equ:Mijk}
	M_{ijk} = n_k c_{kji} = N_{kl} C_{lji} = (NC)^T = C^T N^T = C^T N
\end{equation}
dove
\[
	C^T = (C^T)_{ijk} = C_{kji}
\]
quindi
\[
	E_i = C^T N Y
\]

Abbiamo poi la (\ref{equ:valueRate})
\[
	v = V_i (P_i - E_i)
	= V \left( N Q \delta - C^T N \right) Y
\]

Infine la (\ref{equ:normalInterval}) diventa

\begin{equation}
	\sum_j t_{ij} * y_{ij} + z_{i0} = T_{ij} Y{ij} + Z_i = T Y^T + Z = \vec{1}
\end{equation}


\subsubsection{Verifiche}

Verifica (\ref{equ:Aijk})

\[
\begin{array}{ll}
P_1 = A_{111}Y_{11} + A_{112}Y_{21} +
A_{121}Y_{12} + A_{122}Y_{22} \\
P_2 = A_{211}Y_{11} + A_{212}Y_{21} +
A_{221}Y_{12} + A_{222}Y_{22} \\

A_{111} = n_1 q_{11} \\
A_{112} = n_2 q_{21} \\
A_{121} = 0 \\
A_{122} = 0 \\
A_{211} = 0 \\
A_{212} = 0 \\
A_{221} = n_1 q_{12} \\
A_{222} = n_2 q_{22} \\

P_1 = n_1 q_{11}Y_{11} + n_2 q_{21}Y_{21} \\
P_2 = n_1 q_{12}Y_{12} + n_2 q_{22}Y_{22} \\
\end{array}
\]
equivalente alla (\ref{equ:p_j})
\[
\begin{array}{ll}
p_1 = n_1 q_{11} y_{11} + n_2 q_{21} y_{21} \\
p_2 = n_1 q_{12} y_{12} + n_2 q_{22} y_{22}
\end{array}
\]
cvd.


Verifica (\ref{equ:NQI})
\[
\begin{array}{ll}
A_{ijk} = \delta_{ijl} B_{lk}
= \delta_{ijl} N_{km} Q_{ml} \\

A_{111} = \delta_{111} N_{11} Q_{11} + 
\delta_{111} N_{12} Q_{21}
\delta_{112} N_{11} Q_{12} + 
\delta_{112} N_{12} Q_{22} =
N_{11} Q_{11} \\
A_{112} = \delta_{111} N_{21} Q_{11} + 
\delta_{111} N_{22} Q_{21}
\delta_{112} N_{21} Q_{12} + 
\delta_{112} N_{22} Q_{22} =
N_{22} Q_{21} \\
A_{121} = \delta_{121} N_{11} Q_{11} + 
\delta_{121} N_{12} Q_{21}
\delta_{122} N_{11} Q_{12} + 
\delta_{122} N_{12} Q_{22} = 0 \\
A_{122} = \delta_{121} N_{21} Q_{11} + 
\delta_{121} N_{22} Q_{21}
\delta_{122} N_{21} Q_{12} + 
\delta_{122} N_{22} Q_{22} = 0 \\
A_{211} = \delta_{211} N_{11} Q_{11} + 
\delta_{211} N_{12} Q_{21}
\delta_{212} N_{11} Q_{12} + 
\delta_{212} N_{12} Q_{22} = 0 \\
A_{212} = \delta_{211} N_{21} Q_{11} + 
\delta_{211} N_{22} Q_{21}
\delta_{212} N_{21} Q_{12} + 
\delta_{212} N_{22} Q_{22} = 0 \\
A_{221} = \delta_{221} N_{11} Q_{11} + 
\delta_{221} N_{12} Q_{21}
\delta_{222} N_{11} Q_{12} + 
\delta_{222} N_{12} Q_{22} =
N_{11} Q_{12} \\
A_{222} = \delta_{221} N_{21} Q_{11} + 
\delta_{221} N_{22} Q_{21}
\delta_{222} N_{21} Q_{12} + 
\delta_{222} N_{22} Q_{22} =
N_{22} Q_{22}
\end{array}
\]
equivalente alla (\ref{equ:Aijk}), cvd.

\subsection{Altro}

Let us normalize the problem such that
\[
\left\{
\begin{array}{ll}
	\min C^T X \\
	A_1 X = B_1 \\
	A_2 X <= B_2 \\
	X >= 0
\end{array}
\right.
\]
so
\[
\left\{
\begin{array}{ll}
	\min (-v) = -V \left( N Q \delta - C^T N \right) Y \\
	T Y^T + Z = \vec{1} \\
	-V \left( N Q \delta - C^T N \right) Y \le \vec{0} \\
	Y \ge 0 \\
	Z \ge 0
\end{array}
\right.
\]



\end{document}