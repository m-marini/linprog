\documentclass[a4paper,11pt]{article}
\usepackage[T1]{fontenc}
\usepackage[utf8]{inputenc}
\usepackage{lmodern}

\title{Linear programming notes}
\author{Marco Marini}

\begin{document}

\maketitle
\tableofcontents

\begin{abstract}
This document contains notes about Linear programming.
\end{abstract}

\section{Supply chain model}

We define a simplified supply chain model as a system
that produces products with a chain of product transformations performed by producers.

Let be
\begin{description}
	\item [$ A = a_1 \dots a_n $] the set of producer types
	\item [$ B = b_1 \dots b_n $] the set of product types.
	\item [$  N_i , \;i \in A $] the number of producers of type $ i $.
\end{description}

The system is constrained by some rules:
\begin{enumerate}
	\item the producer can perform only transformations
	matching the producer type.
	\item the producer can perform only a single
	transformation for the its duration.
\end{enumerate}

For each product $ i $ let define the production rule
\[ R_i = (P_i, V_i, Q_i, T_i, C_{i,i}) \]
with
\begin{description}
	\item[$ P_i \in A $ ] the producer
	\item[$ V_i $ ] the value of product
	\item[$ Q_i $ ] the quantity of outcome in duration interval
	\item[$ T_i $ ] the duration interval
	\item[$ C_{i,j} $ ] the quantity of product $ j $ consumed to produce the product $ i $  in duration interval.
\end{description}

\subsection{Production indicators}

By now we do not consider the constraint on the availability of consuming products. It will be considered later.

When a producer of type $ i $ is ready to produce we assign a production slot for the product $ j $ such that
\begin{description}
	\item[ $ Y_{ij} $ ] is the weighted distribution
\end{description}

Obviously the distribution of product not produceable by the producer is zero:
\begin{equation}
	\label{equ:distribution}
	Y_{ij} = 0, \forall j \; | \; P_j \ne i
\end{equation}

The total time of usage of the producer is
\begin{equation}
	Z_i = \sum_{j \in B} Y_{ij} T_j
\end{equation}

During this interval the producer $ i $ produces the product $ j $ for the perid
	\[ Y_{ij} T_j \]

The produced quantity is
	\[ Q_j Y_{ij} \]
	
The production frequency of product $ j $ for all producers of type $ P_i $ is
\begin{equation}
	\label{equ:prodFreq}
	F_i = N_j Q_i \frac{Y_{ji}}{Z_j}, \; j = P_i
\end{equation}

The production value flow of product $ j $ is 
\begin{equation}
	I_i = V_i F_i
\end{equation}

\subsection{Constraints}

Il grafo di produzione è un grafo orientato dove
ogni nodo rappresenta un prodotto e ogni arco la dipendenza tra prodotti consumati e prodotto generato.

Come calcoliamo la produzione totale dei beni considerando le diverse velocità di produzione dei beni?

Dalla (\ref{equ:prodFreq}) sappiamo con che frequenza viene generato un prodotto.

Possiamo quindi calcolare con che frequenza un prodotto viene consumato
\begin{equation}
	E_i = \sum_{j \in B} F_j \frac{C_{ji}}{Q_j}
\end{equation}

Questa frequenza deve essere ovviamente inferiore alla frequenza di produzione quindi

\begin{equation}
	E_i \le F_i
\end{equation}

\end{document}