\documentclass[a4paper,11pt]{article}
\usepackage[T1]{fontenc}
\usepackage[utf8]{inputenc}
\usepackage{lmodern}
\usepackage{verbatim}

\title{Linear programming notes}
\author{Marco Marini}

\begin{document}

\maketitle
\tableofcontents

\begin{abstract}
This document contains notes about Linear programming.
\end{abstract}

\section{Supply chain model}

We define a simplified supply chain model as a system
that produces products with a chain of product transformations performed by suppliers.

Let it be
\begin{description}
	\item [$ A = a_1 \dots a_n $]
		the set of supplier types
	\item [$ B = b_1 \dots b_m $]
		the set of product types.
\end{description}

The supplier can perform only a single transformation for the its duration.
Each product can be produced by a single supplier type.

Let it be
\begin{description}
	\item [$ S = s_i \in A $]
		the supplier of product $ i \in B $
	\item [$ N = n_i $]
		the number of suppliers of type $ i \in A $.
	\item[ $ V = v_{i} $ ]
		the value of product $ i \in B $
	\item[ $ Q = q_{i} $ ]
		the quantity of produced product $ i \in B  $l
	\item[ $ T = t_{i} $ ]
		the production interval for product $ i \in B  $
	\item[$ D = d_{ij} $ ]
		the quantity of product $ j \in B $ consumed to produce the product $ i \in B $.
\end{description}


\subsection{Production}

By now we do not consider the constraint on the availability of consuming products. It will be considered later.

Let be
\begin{description}
	\item [$ \theta_{ij} $]
		The matrix that map product $ j \in B $ to the supplier $ i \in A $
	\item [$ \nu i $]
	The number of supplier of product $ i \in B $
\end{description}
such that
\[
\begin{array}{l}
	\theta_{ij} = 1, s_j = i \\
	\theta_{ij} = 0, s_j \ne i \\
	\nu _i = n_{s_j}
\end{array}
\]

When a supplier of type $ i \in A $ is ready to produce we assign a production slot for the product $ j \in B $ such that $ s_j = i $ and the production time is $ t_j w_j $ where
\begin{description}
	\item[ $ W = w_i $ ] is the slot weight for product $ i \in B $
\end{description}
such that
\begin{equation}
\label{equ:YConstraints}
	w_i \ge 0
\end{equation}

Moreover we may put the supplier $ i \in A $ in idle state for a while
\begin{description}
	\item[$ Z = z_i $]
		is the idle time during a whole production cycle
\end{description}
such that
\begin{equation}
\label{equ:ZConstraints}
	z_i \ge 0
\end{equation}

The total time of production cycle for supplier $ i $ is
\[
\begin{array}{l}
	U_i(w_j, z_i) = \sum_{j \in B | s_j = i} t_j w_j + z_i \\
		= \sum_{j \in B} \theta_{ij} t_j  w_j + z_i \\
		= \sum_{j, k \in B} \theta_{ik} T_{kj} w_j + z_i \\
		= U W + Z
\end{array}
\]
such that
\[
	U = u_{ij} = \sum_{k \in B} \theta_{ik} T_{kj} = \Theta T
\]

Let normalize the total time production to 1
\begin{equation}
\label{equ:normalInterval}
	U_i(w_j, z_i) = 1
\end{equation}

During this interval the suppliers produce the product $ i \in A $ at a slot rate
\begin{equation}
\label{equ:prodFreq}
	R_i(w_i) = n_{s_i} \frac{w_i}{y_{s_i}} = n_{s_i} w_i
	= \sum{j \in B} N_{ij} w_j = N W
\end{equation}

The production rate of product $ i \in B $ is
\begin{equation}
\label{equ:prodRate}
	P_i(w_j) = R_i(w_j) q_i
		= Q_{ik} R_k(w_j)
		= Q N W
		= P W
\end{equation}
such that
\[
	P = Q N
\]

\subsection{Consumption}

The (\ref{equ:prodFreq}) expresses the slot rate of a product for specific suppliers.
Therefore we can calculate the consumption rate of a product $ i \in B$ as the sum of consumptions among to produced products
\begin{equation}
\label{equ:consumRate}
	C_i(w_j) = \sum_{k \in B} R_k(w_j) d_{ki}
			= \sum_{k \in B} d'_{ik} R_k(w_j)
			= D' N W
			= C W
\end{equation}
such that
\[
	C = D' N
\]

The effective production rate is
\[
	F_i(w_j) = P_i(w_j) - C_i(w_j) = (P - C) W = F W
\]
such that
\[
	F = f_{ij} = P - C = (Q - D') N
\]

If
	\[ F_i(w_j) \ge 0 \]
the product is produced at a higher rate then it is consumped creating a surplus that can be sold.

On the other hane we cannot have 
	\[ F_i(w_j) < 0 \]
because it cannot consume more product than the produced.

So the system must satisfy the constraint
\begin{equation}
\label{equ:consumptionConstraint}
	F_i(w_j) \ge 0
\end{equation}
	
\subsection{Profit rate}

The profit rate for the whole supply chain are
\begin{equation}
\label{equ:valueRate}
	G(w_i) =  \sum_{j \in B} v_j F_j(w_i)
	= V' F W
	= G' W
\end{equation}
such that
\[
	G = g_i = (V' F)' ) F' V = N' (Q'-D) V
\]

The problem is to find the optimal production configuration given by the
values of $ w_i $ and $ z_i $ that maximize the value rate $ G(w_i) $.
This is defined by the linear system composed by
(\ref{equ:YConstraints})
(\ref{equ:ZConstraints}),
(\ref{equ:normalInterval}),
(\ref{equ:consumptionConstraint}),
(\ref{equ:valueRate})
\begin{equation}
\label{equ:linsist}
\left\{
\begin{array}{ll}
\max_{(w_i, z_j)} G(w_i) & , i \in B, j \in A \\
U_i(w_j, z_i) = 1 &  , i \in A, j \in B \\
F_i \ge 0 & , i \in B \\
w_i \ge 0 & , i \in B \\
z_i \ge 0 & , i \in A
\end{array}
\right.
\end{equation}


\subsection{Supplier configuration}

Una volta risolto il sistema (\ref{equ:linsist}) ci troviamo con le distribuzioni degli slot di produzione dei produttori.
Per l'uso pratico dobbiamo trasformare i pesi di distribuzione degli slot
in una configurazione di produzione reale.

Il primo passo per far questo è trasformare le $ w_i $ e $ z_i $ nel numero di produttori per prodotto.

Il numero di produttori che producono il prodotto $ i \in B $ è dato dal tempo totale di produzione dei suppliers per ogni singolo prodotto rispetto al totale $ U_i(w_i, z_j) = 1 $ 
\[
	\pi_i = n_{s_i} w_i t_i
\]
\[
  \Pi = N T W
\]

La parte intera di $ \Pi $ ci da il numero di produttori fissi per
prodotto
\[
	F = |\Pi|
\]

Il numero totale di produttori fissi è quindi
\[
	tot_i = \sum_{j \in B}  \theta_{ij} f_j
\]
\[
	Tot = \Theta F
\]

I rimanenti produttori
\[
	Var = N - Tot
\]
sono distribuiti secondo la parte frazionaria $ \Pi - F $.


\end{document}